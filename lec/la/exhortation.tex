Most of mathematics is best learned by doing.  Linear algebra is no
exception.  You have had a previous class in which you learned the
basics of linear algebra, and you will have plenty of practice with
these concepts over the semester.  This brief refresher lecture is
supposed to remind you of what you've already learned and introduce a
few things you may not have seen.  It also serves to set notation
that will be used throughout the class.

In addition to these notes, you may find it useful to go back to a
good linear algebra text (there are several listed on the course
syllabus) or to look at the linear algebra review chapters in the
book.
