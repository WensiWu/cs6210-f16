\section{Basic notational conventions}

In this section, we set out some basic notational conventions used
in the class.
\begin{enumerate}
\item
  The complex unit is $\i$ (not $i$ or $j$).

\item
  By default, all spaces in this class are finite dimensional. If there
  is only one space and the dimension is not otherwise stated, we use
  $n$ to denote the dimension.

\item
  Concrete real and complex vector spaces are $\bbR^n$ and $\bbC^n$,
  respectively.

\item
  Real and complex matrix spaces are $\bbR^{m\times n}$ and $\bbC^{m
  \times n}$.

\item Unless otherwise stated, a concrete vector is a column vector.

\item The vector $e$ is the vector of all ones.

\item The vector $e_i$ has all zeros except a one in the $i$th place.

\item
  The basis $\{ e_i \}_{i=1}^n$ is the {\em standard basis} in $\bbR^n$
  or $\bbC^n$.

\item
  We use calligraphic math caps for abstract space,
  e.g.~$\mathcal{U}, \mathcal{V}, \mathcal{W}$.

\item
   When we say $U$ is a basis for a space $\mathcal{U}$, we mean $U$ is
   an isomorphism $\mathcal{U} \rightarrow \bbR^n$.  By a slight abuse
   of notation, we say $U$ is a matrix whose columns are the abstract
   vectors $u_1, \ldots, u_n$, and we write the linear combination
   $\sum_{i=1}^n u_i c_i$ concisely as $Uc$.

\item
  Similarly, $U^{-1} x$ represents the linear mapping from the
  abstract vector $x$ to a concrete coefficient vector $c$ such
  that $x = Uc$.

\item The space of univariate polynomials of degree at most
  $d$ is $\mathcal{P}_d$.

\item
  Scalars will typically be lower case Greek, e.g.~$\alpha, \beta$. In
  some cases, we will also use lower case Roman letters, e.g.~$c, d$.

\item
  Vectors (concrete or abstract) are denoted by lower case Roman,
  e.g.~$x, y, z$.

\item
  Matrices and linear maps are both denoted by upper case Roman,
  e.g.~$A, B, C$.

\item
  For $A \in \bbR^{m \times n}$, we denote the entry in row $i$ and
  column $j$ by $a_{ij}$.  We reserve the notation $A_{ij}$ to refer to
  a submatrix at block row $i$ and block column $j$ in a partitioning of
  $A$.

\item
  We use a superscript star to denote dual spaces and dual vectors; that
  is, $v^* \in \mathcal{V}^*$ is a dual vector in the space dual to
  $\mathcal{V}$.

\item
  In $\bbR^n$, we use $x^*$ and $x^T$ interchangeably for the transpose.

\item
  In $\bbC^n$, we use $x^*$ and $x^H$ interchangeably for the conjugate
  transpose.

\item
  Inner products are denoted by angles, e.g.~$\ip{x}{y}$. To denote an
  alternate inner product, we use subscripts, e.g. $\ip{x}{y}_M = y^* M
  x$.

\item
  The standard inner product in $\bbR^n$ or $\bbC^n$ is also $x \cdot y$.

\item
  In abstract vector spaces with a standard inner product, we use $v^*$
  to denote the dual vector associated with $v$ through the inner
  product, i.e.~$v^* = (w \mapsto \ip{w}{v})$.

\item
  We use the notation $\|x\|$ to denote a norm of the vector $x$. As
  with inner products, we use subscripts to distinguish between multiple
  norms.  When dealing with two generic norms, we will sometimes use the
  notation $\vertiii{y}$ to distinguish the second norm from the first.

\item
  We use order notation for both algorithm scaling with parameters going
  to infinity (e.g.~$\order{n^3}$ time) and for reasoning about scaling
  with parameters going to zero (e.g.~$\order{\epsilon^2}$ error). We
  will rely on context to distinguish between the two.

\item
  We use {\em variational notation} to denote derivatives of matrix
  expressions, e.g.~$\delta (AB) = \delta A \, B + A \, \delta B$ where
  $\delta A$ and $\delta B$ represent infinitesimal changes to the
  matrices $A$ and $B$.

\item
  Symbols typeset in Courier font should be interpreted as \matlab\ code
  or pseudocode, e.g.~{\tt y = A*x}.

\item
  The function notation $\fl(x)$ refers to taking a real or complex
  quantity (scalar or vector) and representing each entry in floating
  point.

\end{enumerate}
