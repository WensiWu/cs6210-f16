\section{Linear algebra references}

We assume some mathematical pre-requisites: linear algebra
and ``sufficient mathematical maturity.''  This should be enough for
most students, but some will want to brush up on their linear algebra.
For those who want to review, or who want a reference while taking
the course, I have a few recommendations.

There are a few options when it comes to linear algebra basics.  At
Cornell, our undergraduate linear algebra course uses the text
by Lay~\cite{Lay:2016:Linear}; the texts by Strang~\cite{Strang:2006:Linear,Strang:2009:Introduction} are a nice
alternative.  Strang's {\em Introduction to Linear Algebra}~\cite{Strang:2009:Introduction} is the textbook for the MIT
linear algebra course that is the basis for his enormously popular
video lectures, available on MIT's OpenCourseWare site; if you prefer
lecture to reading, Strang is known as an excellent lecturer.

These notes reflect the way that I teach matrix computations, but this
is far from the only approach.  The style of this class is probably
closest to the style of the text by Demmel~\cite{Demmel:1997:Applied},
which I recommend as a course text.  The text of Trefethen and
Bau~\cite{Trefethen:1997:Numerical} covers roughly the same material,
but with a different emphasis; I recommend it for an alternative
perspective.  At Cornell, our subscription to the SIAM book series
means that both of these books are available as e-books to students
on the campus network.

For students who intend to use matrix computations as a serious part of
their future research career, the canonical reference is {\em Matrix
Computations} by Golub and Van Loan~\cite{Golub:2013:Matrix}.
